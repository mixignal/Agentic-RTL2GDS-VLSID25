\documentclass[conference]{IEEEtran}
\IEEEoverridecommandlockouts
% The preceding line is only needed to identify funding in the first footnote. If that is unneeded, please comment it out.
\usepackage{cite}
\usepackage{amsmath,amssymb,amsfonts}
\usepackage{algorithmic}
\usepackage{graphicx}
\usepackage{textcomp}
\usepackage{xcolor}
\def\BibTeX{{\rm B\kern-.05em{\sc i\kern-.025em b}\kern-.08em
    T\kern-.1667em\lower.7ex\hbox{E}\kern-.125emX}}
\begin{document}

\title{Conference Paper Title}

\author{\IEEEauthorblockN{Aditya Patra}
\IEEEauthorblockA{\textit{Department} \\
\textit{School Name}\\
Bay Area, California \\
email address}
\and
\IEEEauthorblockN{Saroj Rout}
\IEEEauthorblockA{\textit{Department of Electronics Engineering} \\
\textit{Silicon University}\\
Bhubaneswar, India \\
0000-0002-5191-8191 }
\and
\IEEEauthorblockN{Arun Ravindran}
\IEEEauthorblockA{\textit{Dept. of Electrical and Computer Engineering} \\
\textit{University of North Carolina Charlotte}\\
Charlotte, USA \\
email/ORCID }

\maketitle

\begin{abstract}
	Abstract here...
\end{abstract}

\begin{IEEEkeywords}
 Agentic Flow, Generative AI, Keyword Spotting (KWS)
\end{IEEEkeywords}

\section{Introduction}
 Intro...

\section*{Acknowledgment}


\section*{References}

\begin{thebibliography}{00}
\bibitem{b1} G. Eason, B. Noble, and I. N. Sneddon, ``On certain integrals of Lipschitz-Hankel type involving products of Bessel functions,'' Phil. Trans. Roy. Soc. London, vol. A247, pp. 529--551, April 1955.
\bibitem{b2} J. Clerk Maxwell, A Treatise on Electricity and Magnetism, 3rd ed., vol. 2. Oxford: Clarendon, 1892, pp.68--73.
\end{thebibliography}

\end{document}
