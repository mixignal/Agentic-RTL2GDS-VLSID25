\section{Related Work (Arun)}
\label{sec:related_work}
We review previous reported work on the use of Generative AI in chip design. We also briefly review the use of agentic Generative AI in software design.

\subsection{Generative AI in Chip Design}
DAVE \cite{dave}Verigen \cite{verigen} are among the first works that seek to leverage the use of LLMs in HDL generation. Their focus is on improving LLM performance by fine-tuning the open source CodeGen-16B model for Verilog code generation. The ChipNemo \cite{chipnemo} project from Nvidia does not directly target the use of LLMs in the design of digital systems. Instead, domain-specific LLMs are designed for supporting tasks such as a chatbot that serves as an engineering assistant, EDA script generation, and bug summarization and analysis. Their efforts are thus complementary to our proposed work. Recognizing the need for augmenting and fine-tuning LLMs for hardware design, the MG-Verilog \cite{mg-verilog} project proposes an open source dataset consisting of over 11,000 Verilog code samples and their corresponding natural language descriptions. MG-Verilog dataset is complementary to our effort, and could be used to make the underlying foundational models more accurate through techniques such as Retrieval Augmented Generation (RAG), and fine tuning.  In the GPT4AIChip project \cite{gpt4aigchip}, a feedback design loop is proposed consisting of prompting the LLM with human crafted design examples (few-shot learning), and an evolutionary algorithm based design space exploration algorithm to explore the designs generated by the LLM.  The LLM generated code for a matrix multiplier (GEMM) is synthesized and evaluated on an FPGA using the Xilinx Vivado HLS tools. The AI design flow proposed by GPT4AIChip can be considered as agentic, their design exploration space is limited, and unlike us it does not target the more complex ASIC design flow with time analysis. The AutoChip \cite{autochip} project proposes an automated approach that uses large language models (LLMs) to generate HDL. AutoChip combines conversational LLMs such as GPT4 with feedback from Verilog compilers and simulations to iteratively improve Verilog modules. Starting with an initial module generated from a design prompt, it refines the design based on errors and simulation messages. AutoChip's effectiveness is evaluated using design prompts and test benches from HDLBits, a collection of small circuit design exercises. While AutoChip can be considered as using an agentic design flow, unlike our system design emphasis, they are limited to small circuits and lack of an end-to-end design flow. 

\end{itemize}
\cite{thakur2024verigen}
 
